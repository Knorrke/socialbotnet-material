%!TEX root = objekteHandout.tex
\documentclass[parskip=half*]{scrartcl}
\setkomafont{section}{\LARGE}
\usepackage{geometry}
\geometry{a4paper, top=2.5cm, left=2cm, right=2cm, bottom=2.5cm}
\pagestyle{empty}
\usepackage[utf8]{inputenc}
\usepackage[T1]{fontenc}
\usepackage[ngerman]{babel}
\usepackage{amsmath}
\usepackage[autostyle=true, german=quotes]{csquotes}
\usepackage{hyperref}
\usepackage{graphicx}
\usepackage{todonotes}
\usepackage{mdframed}
\usepackage{changepage}
\usepackage{tcolorbox}
\usepackage{tabularx}
\newcolumntype{L}[1]{>{\raggedright\arraybackslash}p{#1}} % linksbündig mit Breitenangabe
\newcolumntype{C}[1]{>{\centering\arraybackslash}p{#1}} % zentriert mit Breitenangabe
\newcolumntype{R}[1]{>{\raggedleft\arraybackslash}p{#1}} % rechtsbündig mit Breitenangabe

%% Code
\usepackage{listings}
\usepackage{color}

\definecolor{dkgreen}{rgb}{0,0.6,0}
\definecolor{dkred}{rgb}{0.6,0,0}
\definecolor{gray}{rgb}{0.5,0.5,0.5}
\definecolor{mauve}{rgb}{0.58,0,0.82}

\lstset{frame=tb,
  language=Java,
  aboveskip=3mm,
  belowskip=3mm,
  showstringspaces=false,
  columns=flexible,
  basicstyle={\small\ttfamily},
  numbers=left,
  numberstyle=\tiny\color{gray},
  keywordstyle=\color{blue},
  commentstyle=\color{dkgreen},
  stringstyle=\color{mauve},
  breaklines=true,
  breakatwhitespace=true,
  tabsize=3
}

\begin{document}

\section*{Parsen der Antworten des SocialBotNet}
In dem Projekt steht dir eine Klasse \lstinline{AntwortParser} zur Verfügung.
\begin{figure}[htb]
    \centering
    \begin{tabular}[t]{|l|}
    \hline
    \multicolumn{1}{|c|}{\bfseries AntwortParser} \\ \hline
                    \\ \hline
    zuUserArray(String serverAntwort) : User[ ] \\
    zuPostArray(String serverAntwort) : Post[ ] \\ \hline
    \end{tabular} \\ \vspace{0.1cm}
    \label{fig:klasse_AntwortParser}
\end{figure}

Diese kann die Antworten des SocialBotNet zu richtigen Objekten umwandeln:
\begin{itemize}
\item \lstinline{zuUserArray} wandelt die Antwort (z.B. von \emph{/api/users}) in ein Array von User Objekten.
\item \lstinline{zuPostArray} wandelt die Antwort (z.B. von \emph{/api/posts}) in ein Array von Post Objekten.
\end{itemize}

\begin{figure}[htb]
    \centering
    \begin{tabular}[t]{|l|}
    \hline
    \multicolumn{1}{|c|}{\bfseries User} \\ \hline
     - id : int  \\
     - username : String  \\
     - hobbies : String  \\
     - about : String  \\  \\ \\ \hline
     getId() : int  \\
     getUsername() : String  \\
     getHobbies() : String  \\
     getAbout() : String  \\  \\ \\ \hline
    \end{tabular}
    \hspace{0.8cm}
    \begin{tabular}[t]{|l|}
    \hline
    \multicolumn{1}{|c|}{\bfseries Post} \\ \hline
     - id : int  \\
     - message : String \\
     - publishingDate : Timestamp  \\
     - user : User  \\
     - wall : User  \\
     - trendingScore: double \\
     - likes: int \\
     - likedBy : User[ ]  \\ \hline
     getId() : int  \\
     getMessage() : String  \\
     getPublishingDate() : Timestamp  \\
     ... \\
     getLikedBy() : User[ ] \\ \hline
    \end{tabular}
    \label{fig:klasse_User_Post}
\end{figure}

Beispiel:
\begin{lstlisting}
public void eigenePinnwandLiken() {
    // Alle Nachrichten auf der Pinnwand von MeinBot abrufen
    String antwort = socialbotnet.GETAnfrageSenden("/api/pinnwand/MeinBot");
    // Die Antwort beschreibt mehrere Posts, verwende also AntwortParser.zuPostArray
    Post[] nachrichten = AntowrtParser.zuPostArray(antwort);
    // Iteriere ueber die einzelnen Nachrichten
    for(int i=0; i<nachrichten.length; i++) {
        Post nachricht = nachrichten[i];
        // Hole die Nachrichten ID aus dem Post Objekt
        int postid = nachricht.getId();
        // Diese ID wird als Eingabewert zum Liken benoetigt
        liken(postid);
    }
}

public void liken(int postid) {
    // POST-Anfrage mit Parametern vorbereiten
    socialbotnet.POSTAnfrageVorbereiten("username", username);
    socialbotnet.POSTAnfrageVorbereiten("password", password);
    socialbotnet.POSTAnfrageVorbereiten("postid", postid);
    // Like schicken
    socialbotnet.POSTAnfrageSenden("/api/like");
}
\end{lstlisting}

\newpage
\section*{Beispielrückgaben des SocialBotNet}

Ein \emph{GET}-Request auf \lstinline{/api/users} liefert folgendes Ergebnis, das mit \lstinline{AntwortParser.zuUserArray} verarbeitet werden kann:
\begin{lstlisting}

[
  {
    "id":1,
    "username":"root",
    "hobbies":"Filme schauen, Fussball spielen",
    "about":"I am root."
  },
  {
    "id":2,
    "username":"Welcome",
    "hobbies":"",
    "about":""
  },
  ...
]
\end{lstlisting}


Ein \emph{GET}-Request auf \lstinline{/api/posts} oder \lstinline{/api/pinnwand/...} liefert folgendes Ergebnis, das mit \\\lstinline{AntwortParser.zuPostArray} verarbeitet werden kann:
\begin{lstlisting}
[
  {
    "id":2,
    "message":"Herzlich Willkommen, root!",
    "user":{"id":2,"username":"Welcome", "hobbies": "...", "about": "..."},
    "wall":{"id":1,"username":"root", "hobbies": "...", "about": "..."},
    "publishingDate":"Oct 25, 2014 10:09:55 AM",
    "likedBy":[
      {"id":1,"username":"root", "hobbies": "...", "about": "..."},
      {"id":10,"username":"Hello", "hobbies": "...", "about": "..."}
    ]
  },
  ...
]
\end{lstlisting}

\subsection*{Aufgabe} Zeichne ein Objektdiagramm für die GET-Anfrage \emph{/api/posts?limit=1}.

\end{document}


