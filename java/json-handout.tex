%!TEX root = jsonHandout.tex
\documentclass[parskip=half*]{scrartcl}
\setkomafont{section}{\LARGE}
\usepackage{geometry}
\geometry{a4paper, top=2.5cm, left=2cm, right=2cm, bottom=2.5cm}
\pagestyle{empty}
\usepackage[utf8]{inputenc}
\usepackage[T1]{fontenc}
\usepackage[ngerman]{babel}
\usepackage{amsmath}
\usepackage[autostyle=true, german=quotes]{csquotes}
\usepackage{hyperref}
\usepackage{graphicx}
\usepackage{todonotes}
\usepackage{mdframed}
\usepackage{changepage}
\usepackage{tcolorbox}
\usepackage{tabularx}
\newcolumntype{L}[1]{>{\raggedright\arraybackslash}p{#1}} % linksbündig mit Breitenangabe
\newcolumntype{C}[1]{>{\centering\arraybackslash}p{#1}} % zentriert mit Breitenangabe
\newcolumntype{R}[1]{>{\raggedleft\arraybackslash}p{#1}} % rechtsbündig mit Breitenangabe

%% Code
\usepackage{listings}
\usepackage{color}

\definecolor{dkgreen}{rgb}{0,0.6,0}
\definecolor{dkred}{rgb}{0.6,0,0}
\definecolor{gray}{rgb}{0.5,0.5,0.5}
\definecolor{mauve}{rgb}{0.58,0,0.82}

\lstset{frame=tb,
  language=Java,
  aboveskip=3mm,
  belowskip=3mm,
  showstringspaces=false,
  columns=flexible,
  basicstyle={\small\ttfamily},
  numbers=left,
  numberstyle=\tiny\color{gray},
  keywordstyle=\color{blue},
  commentstyle=\color{dkgreen},
  stringstyle=\color{mauve},
  breaklines=true,
  breakatwhitespace=true,
  tabsize=3
}

\begin{document}

\section*{JSON-Format}

JSON ist ein Übertragungsformat für Daten, die sich leicht mit dem PC verarbeiten lassen sollen. Dabei gibt es Objekte mit Key-Value Paaren und Arrays von Objekten.

\begin{minipage}[t]{0.45\linewidth}
\textbf{Objekte:}

\begin{lstlisting}[frame=single]
{
  "key": "value",
  "key2": 42,
  "key3": true,
  ...
}
\end{lstlisting}
\end{minipage}
\hspace{0.1\linewidth}
\begin{minipage}[t]{0.45\linewidth}
\textbf{Arrays:}

\begin{lstlisting}[frame=single]
[
  {
    "key": "objekt1",
  },
  {
    "key": "objekt2"
  },
  ...
]
\end{lstlisting}
\end{minipage}
Als Values in den Objekten sind zugelassen: Zahlen, Booleanwerte, Strings, oder ein weiteres Objekt oder sogar ein Array. Somit lassen sich die Objekte auch schachteln.

JSON lässt sich in Java mit folgenden Methoden verarbeiten:
\begin{itemize}
  \item Relevante Methoden von JSONArray: \lstinline{getJSONObject(int index)}
  \item Relevante Methoden von JSONObject: \lstinline{getInt(String key)}, \lstinline{getBoolean(String key)}, \\\lstinline{getString(String key)}, \lstinline{getJSONObject(String key)}, \lstinline{getJSONArray(String key)}
\end{itemize}

Beispiel:

\begin{lstlisting}
// Erstellt aus dem String ein JSONArray
JSONArray array = new JSONArray("[{ \"id\": 1 }]");

// Mit getJSONObject(int index) kann man auf die Objekte im Array zugreifen
JSONObject obj = array.getJSONObject(0);

// Mit getInt(String key) kann man fuer einen Key den Value auslesen
int zahl = obj.getInt("id");
\end{lstlisting}

Beispiel im SocialBotNet
\begin{lstlisting}
public void eigenePinnwandLiken() {
    // Alle Nachrichten auf der Pinnwand von MeinBot abrufen
    String antwort = socialbotnet.GETAnfrageSenden("/api/pinnwand/MeinBot");
    // Die Antwort ist ein JSONArray der Nachrichten
    JSONArray nachrichten = new JSONArray(antwort);
    // Iteriere ueber die einzelnen Nachrichten
    for(int i=0; i<nachrichten.length(); i++) {
        JSONObject nachricht = nachrichten.getJSONObject(i);
        // Hole die Nachrichten ID aus dem JSONObject
        int postid = nachricht.getInt("id");
        // Diese ID wird als Eingabewert zum Liken benoetigt
        liken(postid);
    }
}

public void liken(int postid) {
  // ... POST-Anfrage auf /api/like mit gesetzter postid
}
\end{lstlisting}

\newpage
\section*{Beispielrückgaben des SocialBotNet}

Ein \emph{GET}-Request auf \lstinline{/api/users} liefert folgendes Ergebnis:
\begin{lstlisting}

[
  {
    "id":1,
    "username":"root",
    "hobbies":"Filme schauen, Fussball spielen",
    "about":"I am root."
  },
  {
    "id":2,
    "username":"Welcome",
    "hobbies":"",
    "about":""
  },
  ...
]
\end{lstlisting}


Ein \emph{GET}-Request auf \lstinline{/api/users} oder \lstinline{/api/pinnwand/...} liefert folgendes Ergebnis:
\begin{lstlisting}
[
  {
    "id":2,
    "message":"Herzlich Willkommen, root!",
    "user":{"id":2,"username":"Welcome", "hobbies": "...", "about": "..."},
    "wall":{"id":1,"username":"root", "hobbies": "...", "about": "..."},
    "publishingDate":"Oct 25, 2014 10:09:55 AM",
    "likedBy":[
      {"id":1,"username":"root", "hobbies": "...", "about": "..."},
      {"id":10,"username":"Hello", "hobbies": "...", "about": "..."}
    ]
  },
  ...
]
\end{lstlisting}

\end{document}


