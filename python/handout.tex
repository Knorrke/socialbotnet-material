%!TEX root = handout.tex
\documentclass[parskip=half*]{scrartcl}
\setkomafont{section}{\LARGE}
\usepackage{geometry}
\geometry{a4paper, top=2.5cm, left=2cm, right=2cm, bottom=2.5cm}
\pagestyle{empty}
\usepackage[utf8]{inputenc}
\usepackage[T1]{fontenc}
\usepackage[ngerman]{babel}
\usepackage{amsmath}
\usepackage[autostyle=true, german=quotes]{csquotes}
\usepackage{hyperref}
\usepackage{graphicx}
\usepackage{todonotes}
\usepackage{mdframed}
\usepackage{changepage}
\usepackage{tcolorbox}
\usepackage{tabularx}
\newcolumntype{L}[1]{>{\raggedright\arraybackslash}p{#1}} % linksbündig mit Breitenangabe
\newcolumntype{C}[1]{>{\centering\arraybackslash}p{#1}} % zentriert mit Breitenangabe
\newcolumntype{R}[1]{>{\raggedleft\arraybackslash}p{#1}} % rechtsbündig mit Breitenangabe

%% Code
\usepackage{listings}
\usepackage{color}

\definecolor{dkgreen}{rgb}{0,0.6,0}
\definecolor{dkred}{rgb}{0.6,0,0}
\definecolor{gray}{rgb}{0.5,0.5,0.5}
\definecolor{mauve}{rgb}{0.58,0,0.82}

\lstset{frame=tb,
  language=Java,
  aboveskip=3mm,
  belowskip=3mm,
  showstringspaces=false,
  columns=flexible,
  basicstyle={\small\ttfamily},
  numbers=left,
  numberstyle=\tiny\color{gray},
  keywordstyle=\color{blue},
  commentstyle=\color{dkgreen},
  stringstyle=\color{mauve},
  breaklines=true,
  breakatwhitespace=true,
  tabsize=3
}

\begin{document}

\section*{Programmierschnittstellen im Web}

\vspace{0.3cm}

Webseiten bieten Programmierer*innen oft Schnittstellen, sogenannte \emph{API}s. An diese Schnittstellen können Programme Anfragen stellen. Dabei gibt es zwei wichtige Arten:
\begin{itemize}
  \item \textbf{\emph{POST}-Anfragen}, um dem Server Informationen zum Verarbeiten zu \textbf{senden}.
  \item \textbf{\emph{GET}-Anfragen}, um Informationen vom Server \textbf{abzufragen}
\end{itemize}

\subsection*{POST-Anfragen}

\emph{POST}-Anfragen benötigen Daten, die an das Netzwerk geschickt werden müssen. Als Hilfestellung steht die Klasse \lstinline{NetzwerkZugriff} zur Verfügung.
\begin{figure}[htb]
    \centering
    \begin{tabular}{|l|}
    \hline
    \multicolumn{1}{|c|}{\bfseries Netzwerkzugriff} \\ \hline
                    \\ \hline
    NetzwerkZugriff(String domain)                \\
    GETAnfrageSenden(String url) : String \\
    POSTAnfrageSenden(String url, parameter1=String parameterWert, ...) : void \\ \hline
    \end{tabular}
    \label{fig:klasse_netzwerkzugriff}
\end{figure}

Du kannst durch die Methode \lstinline{POSTAnfrageSenden} Werte für deine Anfrage setzen und abgeschicken. Z. B.:

\begin{lstlisting}
socialbotnet = NetzwerkZugriff("https://www.socialbotnet.de")

// POST-Anfrage abschicken
socialbotnet.POSTAnfrageSenden("/api/post", username="MeinBot", password="secure", message="Hallo Welt!")
\end{lstlisting}

\subsection*{GET-Anfragen}
GET-Anfragen können mit Objekten der Klasse \lstinline{NetzwerkZugriff} durch die Methode \lstinline{GETAnfrageSenden} gesendet werden. Z.B.
\begin{lstlisting}
socialbotnet = NetzwerkZugriff("https://www.socialbotnet.de")

// Alle User abfragen
antwort = socialbotnet.GETAnfrageSenden("/api/users")
\end{lstlisting}
\vspace{0.25cm}

Auch bei GET-Anfragen können Parameter benutzt werden. Diese werden jedoch einfach in der URL angehängt:
\begin{lstlisting}
// Posts abfragen, mit den Parametern "sortby=likes" und "limit=1"
antwort = socialbotnet.GETAnfrageSenden("/api/posts?sortby=likes&limit=1")
\end{lstlisting}
\newpage
\section*{Übersicht der Schnittstellen}
\bgroup
\def\arraystretch{1.5}%  1 is the default, change whatever you need
\begin{table}[ht]
\begin{center}
\begin{tabular}{|L{7cm}|C{1.2cm}|L{7cm}|}\hline
\emph{/api/users} & \emph{GET} & Liefert eine Übersicht über alle registrierten Nutzer\\\hline

\emph{/api/posts} & \emph{GET} & Liefert die letzten 50 Posts des gesamten Netzwerks\\ \cline{3-3}
Optionale Parameter: sortby, limit & & \emph{sortby=likes} nach Likes sortiert, \emph{sortby=time} nach Datum sortiert. Limit beschränkt die Anzahl, z.B. limit=1 für nur einen Post\\ \hline

\emph{/api/pinnwand/}\texttt{<username>} & \emph{GET} & Liefert die Posts an der Pinnwand des genannten Nutzers\\\cline{3-3}
Optionale Parameter: sortby, limit & & sortby=likes, sortby=time, limit=zahl wie oben\\ \hline\hline

\emph{/api/user/update} & \emph{POST} & Updaten der Profilinformationen wie Nutzername, \enquote{Hobbies} und \enquote{Über mich}.\\\cline{3-3}
Parameter: username, password

Optionale Parameter: newUsername, hobbies, about & & \emph{newUsername} Benutzernamen ändern, \emph{hobbies} und \emph{about} setzen den \enquote{Hobbies} und \enquote{Über mich} Text.\\ \hline

\emph{/api/post} & \emph{POST} & Posten eines Beitrags auf der eigenen Pinnwand und der allgemeinen Chronik.\\
Parameter: username, password, message & & \\ \hline

\emph{/api/post/}\textbf{\texttt{<username>}} & \emph{POST} & Posten an die Pinnwand eines bestimmten Nutzers. \\
Parameter: username, password, message & &\\\hline

\emph{/api/like} & \emph{POST} & Ermöglicht automatisiertes Liken eines Posts.\\
Parameter: username, password, postid & & \\\hline
\emph{/api/unlike} & \emph{POST} & Macht like rückgängig.\\ 
Parameter: username, password, postid & & \\\hline

\end{tabular}	
\end{center}
\end{table}
\egroup
\end{document}

% {\Large \textbf{JSON-Format}}

% JSON ist ein Übertragungsformat für Daten, die sich leicht mit dem PC verarbeiten lassen sollen. Dabei gibt es Objekte mit Key-Value Paaren und Arrays von Objekten.

% \begin{minipage}[t]{0.45\linewidth}
% \textbf{Objekte:}

% \begin{lstlisting}[frame=single]
% {
%   "key": "value",
%   "key2": 42,
%   "key3": true,
%   ...
% }
% \end{lstlisting}
% \end{minipage}
% \hspace{0.1\linewidth}
% \begin{minipage}[t]{0.45\linewidth}
% \textbf{Arrays:}

% \begin{lstlisting}[frame=single]
% [
%   {
%     "key": "objekt1",
%   },
%   {
%     "key": "objekt2"
%   },
%   ...
% ]
% \end{lstlisting}
% \end{minipage}
% Als Values in den Objekten sind zugelassen: Zahlen, Booleanwerte, Strings, oder ein weiteres Objekt oder sogar ein Array. Somit lassen sich die Objekte auch schachteln.

% JSON lässt sich in Java mit folgenden Methoden verarbeiten:
% \begin{itemize}
% 	\item Relevante Methoden von JSONArray: getJSONObject(int index)
% 	\item Relevante Methoden von JSONObject: getInt(String key), getBoolean(String key), getString(String key), getJSONObject(String key), getJSONArray(String key)
% \end{itemize}

% Beispiel:
% \begin{figure}[htb]
% \centering
% \includegraphics[width=\linewidth]{bilder/json.png}
% \end{figure}
